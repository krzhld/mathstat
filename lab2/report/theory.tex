\section{Теория}

\subsection{Двумерное нормальное распределение}
Двумерная случайная величина $(X, Y)$ называется распределённой нормально (или просто нормальной), если её плотность вероятности определена формулой:
\begin{equation}
	N(x, y, \bar{x}, \bar{y}, \sigma_x, \sigma_y, \rho) = 
	\dfrac{1}{2 \pi \sigma_x \sigma_y \sqrt{1 - \rho^2}} \cdot 
	\exp{\begin{Bmatrix}
			-\frac{1}{2 (1 - \rho^2)}
			\begin{bmatrix}
				\frac{(x - \bar{x})^2}{\sigma_{x}^2} 
				- 2 \rho \frac{(x - \bar{x})(y - \bar{y})}{\sigma_x \sigma_y} 
				+ \frac{(y - \bar{y})^2}{\sigma_{y}^2}
			\end{bmatrix}
		\end{Bmatrix}}
\end{equation}
Компоненты $X, Y$ двумерной нормальной случайной величины также распределены нормально с математическими ожиданиями $\bar{x}, \bar{y}$ и средними квадратическими отклонениями $\sigma_{x}, \sigma_{y}$ соответственно. Параметр $\rho$ называется коэффициентом корреляции.

\subsection{Корреляционный момент (ковариация) и коэффициент корреляции}
Корреляционным моментом, иначе ковариацией, двух случайных величин $X \text{ и } Y$ называется математическое ожидание произведения отклонений этих случайных величин от их математических ожиданий.
\begin{equation}
	K = cov(X, Y) = M[(X - \bar{x})(Y - \bar{y})]
\end{equation}
Коэффициентом корреляции $\rho$ двух случайных величин $X \text{ и } Y$ называется отношение их корреляционного момента к произведению их средних квадратических отклонений:
\begin{equation}
	\rho = \dfrac{K}{\sigma_{x} \sigma_{y}}
\end{equation}
Коэффициент корреляции -- это нормированная числовая характеристика, являющаяся мерой близости зависимости между случайными величинами к линейной.

\subsection{Выборочные коэффициенты корреляции}

\subsubsection{Выборочный коэффициент корреляции Пирсона}
Пусть по выборке значений $\{(x_{i},y_{i})\}_{i=1}^{n}$ двумерной с.в. $(X, Y)$ требуется оценить коэффициент корреляции $\rho$. Естественной оценкой для $\rho$ служит его статистический аналог в виде выборочного коэффициента корреляции, предложенного К.Пирсоном:
\begin{equation}
	r = \dfrac{\frac{1}{n} \sum_{i=1}^{n} (x_i - \bar{x})(y_i - \bar{y})}{\sqrt{\frac{1}{n} \sum_{i=1}^{n} (x_i - \bar{x})^2 \text{ } \frac{1}{n} \sum_{i=1}^{n} (y_i - \bar{y})^2}} = \dfrac{K}{s_X s_Y} \text{,}
\end{equation}
где $K, s_{X}^2, s_{Y}^2$ -- выборочные ковариация и дисперсии с.в. $X \text{ и } Y$.

\subsubsection{Выборочный квадрантный коэффициент корреляции}
Кроме выборочного коэффициента корреляции Пирсона, существуют и другие оценки степени взаимосвязи между случайными величинами. К ним относится выборочный квадрантный коэффициент корреляции:
\begin{equation}
	r_Q = \dfrac{(n_1 + n_3) - (n_2 + n_4)}{n} \text{,}
\end{equation}
где $n_1, n_2, n_3, n_4$ -- количества точек с координатами $x_i, y_i$, попавшими соответственно в \RomanNumeralCaps{1}, \RomanNumeralCaps{2}, \RomanNumeralCaps{3}, \RomanNumeralCaps{4} квадранты декартовой системы координат с осями $x^\prime = x - med \ x, y^\prime = y - med \ y$ и с центром в точке с координатами $(med \ x, med \ y)$.

\subsubsection{Выборочный коэффициент ранговой корреляции Спирмена}
На практике нередко требуется оценить степень взаимодействия между качественными признаками изучаемого объекта. Качественным называется признак, который нельзя измерить точно, но который позволяет сравнивать изучаемые объекты между собой и располагать их в порядке убывания или возрастания их качества. Для этого объекты выстраиваются в определённом порядке в соответствии с рассматриваемым признаком. Процесс упорядочения называется ранжированием, и каждому члену упорядоченной последовательности объектов присваивается ранг, или порядковый номер. Например, объекту с наименьшим значением признака присваивается ранг 1, следующему за ним объекту — ранг 2, и т.д. Таким образом, происходит сравнение каждого объекта со всеми объектами изучаемой выборки. \\
Если объект обладает не одним, а двумя качественными признаками -- переменными $X \text{ и } Y$, то для исследования их взаимосвязи используют выборочный коэффициент корреляции между двумя последовательностями рангов этих признаков. \\
Обозначим ранги, соотвествующие значениям переменной $X$, через $u$, а ранги, соответствующие значениям переменной $Y$ -- через $v$. \\
Выборочный коэффициент ранговой корреляции Спирмена определяется как выборочный коэффициент корреляции Пирсона между рангами $u, v$ переменных $X, Y$:
\begin{equation}
	r_S = \dfrac{\frac{1}{n} \sum_{i=1}^{n} (u_i - \bar{u})(v_i - \bar{v})}{\sqrt{\frac{1}{n} \sum_{i=1}^{n} (u_i - \bar{u})^2 \text{ } \frac{1}{n} \sum_{i=1}^{n} (v_i - \bar{v})^2}} \text{,}
\end{equation}
где $\bar{u} = \bar{v} = \frac{1+2+ \dots + n}{n} = \frac{n+1}{2}$ -- среднее значение рангов.

\subsection{Эллипсы рассеивания}
Рассмотрим поверхность распределения, изображающую функцию \eqref{eq:mix_gauss}. Она имеет вид холма, вершина которого находится над точкой $(\bar{x}, \bar{y})$. \\
В сечении поверхности распределения плоскостями, параллельными оси $N(x, y, \bar{x}, \bar{y}, \sigma_{x}, \sigma_{y}, \rho)$, получаются кривые, подобные нормальным кривым распределения. В сечении поверхности распределения плоскостями, параллельными плоскости $xOy$, получаются эллипсы. Напишем уравнение проекции такого эллипса на плоскость $xOy$:
\begin{equation} \label{eq:ellipse}
	\dfrac{(x - \hat{x})^2}{\sigma_{x}^2} - 2 \rho \dfrac{(x - \hat{x}) (y - \hat{y})}{\sigma_{x} \sigma_{y}} + \dfrac{(y - \hat{y})^2}{\sigma_{y}^2} = Const
\end{equation}
Уравнение эллипса \eqref{eq:ellipse} можно проанализировать обычными методами аналитической геометрии. Применяя их, убеждаемся, что центр эллипса \eqref{eq:ellipse} находится в точке с координатами $(\hat{x}, \hat{y})$. \\
Направления осей симметрии эллипса составляют с осью $Ox$ углы, определяемые уравнением
\begin{equation}
	\tan(2 \alpha) = \dfrac{2 \rho \sigma_{x} \sigma_{y}}{\sigma_{x}^2 - \sigma_{y}^2}
\end{equation}
Это уравнение даёт два значения углов: $\alpha \text{ и } \alpha_1 \text{, различающиеся на } \frac{\pi}{2}$. \\
Таким образом, ориентация эллипса \eqref{eq:ellipse} относительно координатных осей находится в прямой зависимости от коэффициента корреляции $\rho$ системы $(X, Y)$. Если величины не коррелированны (т.е. в данном случае и независимы), то оси симметрии эллипса параллельны координатным осям; в противном случае они составляют с координатными осями некоторый угол. Пересекая поверхность распределения плоскостями, параллельными плоскости $xOy$, и проектируя сечения на плоскость $xOy$, мы получим целое семейство подобных и одинаково расположенных эллипсов с общим центром $(\hat{x}, \hat{y})$. Во всех точках каждого из таких эллипсов плотность распределения $N(x, y, \bar{x}, \bar{y}, \sigma_{x}, \sigma_{y}, \rho)$ постоянна. Поэтому такие эллипсы называются эллипсами равной плотности или, короче эллипсами рассеивания. Общие оси всех эллипсов рассеивания называются главными осями рассеивания.

\subsection{Простая линейная регрессия}

\subsubsection{Модель простой линейной регрессии}
Регрессионную модель описания данных называют простой линейной регрессией, если
\begin{equation} \label{eq:lin_regression}
	y_i = \beta_0 + \beta_1 x_i + \epsilon_i, i=\overline{1, n} \text{, где}
\end{equation}
$x_1, \dots, x_n$ -- заданные числа (значения фактора); \\
$y_1, \dots, y_n$ -- наблюдаемые значения отклика; \\
$\epsilon_1, \dots, \epsilon_n$ -- независимые, нормально распределенные $N(0, \sigma)$ с нулевым математическим ожиданием и одинаковой (неизвестной) дисперсией случайные величины (ненаблюдаемые); \\
$\beta_0, \beta_1$ -- неизвестные параметры, подлежащие оцениванию. \\
В модели \eqref{eq:lin_regression} отклик $y \text { зависит от одного фактора } x$, и весь разброс экспериментальных  точек объясняется только погрешностями наблюдений (результатов измерений) отклика $y$. Погрешности результатов измерений $x$ в этой модели полагают существенно меньшими погрешностей результатов измерений $y$, так что ими можно пренебречь.

\subsubsection{Метод наименьших квадратов}
При оценивании параметров регрессионной модели используют различные методы. Один из наиболее распрстранённых подходов заключается в следующем: вводится мера (критерий) рассогласования отклика и регрессионной функции, и оценки параметров регрессии определяются так, чтобы сделать это рассогласование наименьшим. Достаточно простые расчётные формулы для оценок получают при выборе критерия в виде суммы квадратов отклонений значений отклика от значений регрессионной функции (сумма квадратов остатков):
\begin{equation} \label{eq:least_squares_method}
	Q(\beta_0, \beta_1) = \sum_{i=1}^{n} \epsilon_{i}^2 = \sum_{i=1}^{n} (y_i - \beta_0 - \beta_1 x_i)^2 \rightarrow \min\limits_{\beta_0, \beta_1}
\end{equation}
Задача минимизации квадратичного критерия $Q(\beta_0, \beta_1)$ носит название задачи метода наименьших квадратов (МНК), а оценки $\hat{\beta_0}, \hat{\beta_1}$ параметров $\beta_0, \beta_1$, реализующие минимум критерия $Q(\beta_0, \beta_1)$, называют МНК-оценками.

\subsubsection{Расчётные формулы для МНК-оценок}
МНК-оценки параметров $\hat{\beta_0}, \hat{\beta_1}$ находятся из условия обращения функции \eqref{eq:least_squares_method} в минимум. \\
Для нахождения МНК-оценок $\hat{\beta_0}, \hat{\beta_1}$ выпишем необходимые условия экстремума:
\begin{equation} \label{eq:necessary_conditions_1}
	\begin{cases}
		\frac{\partial Q}{\partial \beta_0} = -2 \sum_{i=1}^{n} (y_i - \beta_0 - \beta_1 x_i) = 0 \\
		\frac{\partial Q}{\partial \beta_1} = -2 \sum_{i=1}^{n} (y_i - \beta_0 - \beta_1 x_i)x_i = 0 
	\end{cases}
\end{equation}
Из системы \eqref{eq:necessary_conditions_1} получим:
\begin{equation}
	\begin{cases} \label{eq:necessary_conditions_2}
		n \hat{\beta_0} + \hat{\beta_1} \sum x_i = \sum y_i \\
		\hat{\beta_0} \sum x_i + \hat{\beta_1} \sum x_{i}^2 = \sum x_i y_i
	\end{cases}
\end{equation}
Разделим \eqref{eq:necessary_conditions_2} на $n$:
\begin{equation} \label{eq:necessary_conditions_3}
	\begin{cases}
		\hat{\beta_0} + \hat{\beta_1} (\frac{1}{n} \sum x_i) = \frac{1}{n} \sum y_i \\
		\hat{\beta_0} (\frac{1}{n} \sum x_i) + \hat{\beta_1} (\frac{1}{n} \sum x_{i}^2) = \frac{1}{n} \sum x_i y_i
	\end{cases}
\end{equation}
Введём обозначения для выборочных первых и вторых начальных моментов:
\begin{equation} \label{eq:def}
	\overline{x} = \frac{1}{n} \sum x_i, \ \overline{y} = \frac{1}{n} \sum y_i, \ \overline{x^2} = \frac{1}{n} \sum x_{i}^2, \ \overline{x y} = \frac{1}{n} \sum x_i y_i
\end{equation}
Из \eqref{eq:necessary_conditions_3} и \eqref{eq:def} получим:
\begin{equation} \label{eq:necessary_conditions_fin}
	\begin{cases}
		\hat{\beta_0} + \hat{\beta_1} \overline{x} = \overline{y} \\
		\hat{\beta_0} \overline{x} + \hat{\beta_1} \overline{x^2} = \overline{x y}
	\end{cases}
\end{equation}
откуда находим МНК-оценку $\hat{\beta_1}$ наклона прямой регрессии по формуле Крамера:
\begin{equation} \label{eq:beta_0}
	\hat{\beta_1} = \dfrac{\overline{x y} - \overline{x} \cdot \overline{y}}{\overline{x^2} - (\overline{x})^2}
\end{equation}
а МНК-оценку $\hat{\beta_0}$ определяем непосредственно из первого уравнения \eqref{eq:necessary_conditions_fin}:
\begin{equation} \label{eq:beta_1}
	\hat{\beta_0} = \overline{y} - \overline{x} \hat{\beta_1}
\end{equation}
Заметим, что определитель системы \eqref{eq:necessary_conditions_fin}:
\begin{equation}
	\overline{x^2} - (\overline{x})^2 = \frac{1}{n} \sum (x_i - \overline{x})^2 = s_{x}^2 > 0 \text{,}
\end{equation}
если среди значений $x_1, \dots, x_n$ есть различные, что и будем предполагать. \\
Доказательство минимальности функции $Q(\hat{\beta_0}, \hat{\beta_1})$ в стационарной точке проведём с помощью достаточного признака экстремума функции двух переменных. Для начала посчитаем все производные второй степени:
\begin{equation}
	\dfrac{\partial^2 Q}{\partial \beta_{0}^2} = 2n, \ 
	\dfrac{\partial^2 Q}{\partial \beta_{1}^2} = 2 \sum x_{i}^2 = 2 n \overline{x^2}, \ 
	\dfrac{\partial^2 Q}{\partial \beta_{1} \partial \beta_{0}} = 2 \sum x_i = 2 n \overline{x}
\end{equation}
Посчитаем определитель квадратичной формы, соответствующей \eqref{eq:least_squares_method}
\begin{equation} \label{eq:det}
	\det = \dfrac{\partial^2 Q}{\partial \beta_{0}^2} \cdot \dfrac{\partial^2 Q}{\partial \beta_{1}^2} - (\dfrac{\partial^2 Q}{\partial \beta_{1} \partial \beta_{0}})^2 = 
	4 n^2 \overline{x^2} - 4 n^2 (\overline{x})^2 =
	4 n^2 [\overline{x^2} - (\overline{x})^2] = 
	4 n^2 [\dfrac{1}{n} \sum (x_i - \overline{x})] = 4 n^2 s_{x}^2 > 0
\end{equation} 
\eqref{eq:det} и $\frac{\partial^2 Q}{\partial \beta_{0}^2} = 2 n > 0 \implies Q \text{ в стационарной точке имеет минимум.}$   

\subsection{Робастные оценки коэффициентов линейной регрессии}
Робастность оценок коэффициентов линейной регрессии (т.е. их устойчивость по отношению к наличию в данных редких, но больших по величине выбросов) может быть обеспечена различными способами. Одним из них является использование метода наименьших модулей вместо метода наименьших квадратов:
\begin{equation} \label{least_modules_method}
	\sum_{i=1}^{n} |y_i - \beta_0 - \beta_1 x_i| \rightarrow \min_{\beta_0, \beta_{1}}
\end{equation}
Использование метода наименьших модулей в задаче оценивания параметра сдвига распределений приводит к оценке в виде выборочной медианы, обладающей робастными свойствами. В отличие от этого случая и от задач метода наименьших квадратов, на практике задача \eqref{eq:least_squares_method} решается численно. Соответствующие процедуры представлены в некоторых современных пакетах программ по статистическому анализу. \\
Здесь мы рассмотрим простейшую в вычислительном отношении робастную альтернативу оценкам коэффициентов линейной регрессии по МНК. Для этого сначала запишем выражения для оценок \eqref{eq:beta_1} и \eqref{eq:beta_0} в другом виде:
\begin{equation} \label{eq:another_betas}
	\begin{cases}
		\hat{\beta_1} =
		\dfrac{\overline{x y} - \overline{x} \cdot \overline{y}}{\overline{x^2} - (\overline{x})^2} = 
		\dfrac{k_{x y}}{s_x s_y} = \dfrac{k_{x y}}{s_x s_y} \cdot \dfrac{s_y}{s_x} = r_xy \dfrac{s_y}{s_x} \\
		\hat{\beta_0} = \overline{y} - \overline{x} \hat{\beta_1}
	\end{cases}
\end{equation}
В формулах \eqref{eq:another_betas} заменим выборочные средние $\overline{x} \text{ и } \overline{y}$ соответственно на робастные выборочные медианы $med \ x \text{ и } \med \ y$, среднеквадратические отклонения $s_x \text{ и } s_y$ на робастные нормированные интерквартильные широты $q_{x}^* \text{ и } q_{y}^*$, выборочный коэффициент корреляции $r_{x y}$ -- на знаковый коэффициент корреляции $r_Q$:
\begin{equation}
	\hat{\beta_1}_R = r_Q \dfrac{q_{y}^*}{q_{x}^*} \text{,}
\end{equation}
\begin{equation}
	\hat{\beta_0}_R = med \ y - \hat{\beta_1}_R med \ x \text{,}
\end{equation}
\begin{equation}
	r_Q = \dfrac{1}{n} \sum_{i=1}^n sgn (x_i - med \ x) \cdot sgn (y_i - med \ y) \text{,}
\end{equation}
\begin{multline}
	\\
	q^{*}_{y} = \frac{y_{(j)} -y_{(l)}}{k_{q}(n)},~~~
	q^{*}_{x} = \frac{x_{(j)} - x_{(l)}}{k_{q}(n)}, \\ 
	\begin{cases}
		& [\frac{n}{4}] + 1 \text{ при } \frac{n}{4} \text{ дробном, } \\ 
		& \frac{n}{4} \text{ при } \frac{n}{4} \text{ целом. }
	\end{cases}\\
	j = n - l + 1\\  
\end{multline}
Уравнение регрессии имеет вид:
\begin{equation} \label{eq:lin_regression_rob}
	y = \hat{\beta_1}_R + \hat{\beta_0}_R x
\end{equation}
Статистики выборочной медианы и интерквартильной широты обладают робастными свойствами в силу того, что основаны на центральных порядковых статистиках, малочувствительных к большим по величине выбросам в данных. Статистика выборочного знакового коэффициента корреляции робастна, так как знаковая функция $sgn(z)$ чувствительна не к величине аргумента, а только к его знаку. Отсюда оценка прямой регрессии \eqref{eq:lin_regression_rob} обладает очевидными робастными свойствами устойчивости к выбросам по координате $y$, но она довольно груба.

\subsection{Метод максимального правдоподобия}
$L(x_1, \dots, x_n, \theta)$ -- функция правдоподобия (ФП), представляющая собой совместную плотность вероятности независимых с.в. $x_1, \dots, x_n$ и рассматриваемая как функция неизвестного параметра $\theta$:
\begin{equation}
	L(x_1, \dots, x_n, \theta) = f(x_1, \theta) f(x_2, \theta) \dots f(x_n, \theta)
\end{equation}
\textbf{\emph{Определение.}}  Оценкой максимального правдоподобия (о.м.п) будем называть такое значение $\hat{\theta_{мп}}}$ из множества допустимых значений параметра $\theta$, для которого ФП принимает наибольшее значение при заданных $x_1, \dots, x_n$:
\begin{equation}
	\hat{\theta_{мп}} = \arg \max_\theta L(x_1, \dots, x_n, \theta)
\end{equation}
Система уравнений правдоподобия (в случае дифференцируемости функции правдоподобия):
\begin{equation}
	\dfrac{\partial L}{\partial \theta_k} = 0 \text{ или } \dfrac{\partial \ln L}{\partial \theta_k} = 0, k = 1, \dots, m
\end{equation}

\subsection{Проверка гипотезы о законе распределения генеральной совокупности. Метод хи-квадрат}
Исчерпывающей характеристикой изучаемой случайной величины является её закон распределения. Поэтому естественно стремление исследователей построить этот закон приближённо на основе статистических данных. \\
Сначала выдвигается гипотеза о виде закона распределения. \\
После того как выбран вид закона, возникает задача оценивания его параметров и проверки (тестирования) закона в целом. \\
Для проверки гипотезы о законе распределения применяются критерии согласия. Таких критериев существует много. Мы рассмотрим наиболее обоснованный и наиболее часто используемый в практике — критерий $\chi^2$ (хи-квадрат), введённый К.Пирсоном (1900 г.) для случая, когда параметры распределения известны. Этот критерий был существенно уточнён Р.Фишером (1924 г.), когда параметры распределения оцениваются по выборке, используемой для проверки. \\
Ограничимся рассмотрением случая одномерного распределения. \\
Итак, выдвинута гипотеза $H_0$ о генеральном законе распределения с функцией распределения $F(x)$. \\
Рассматриваем случай, когда гипотетическая функция распределения $F(x)$ не содержит неизвестных параметров. \\
Разобьём генеральную совокупность, т.е. множество значений изучаемой случайной величины $X$ на $k$ непересекающихся подмножеств $\Delta_1, \Delta_2, \dots, \Delta_k$. \\
Пусть $p_i = P (X \in \Delta_i), i = 1, \dots, k$. \\
Если генеральная совокупность -- вся вещественная ось, то подмножества $\Delta_i = (a_{i-1}, a_i]$ -- полуоткрытые промежутки $(i = 2, \dots, k - 1)$. Крайние промежутки будут полубесконечными: $\Delta_1 = (-\infty, a_1), \Delta_k = (a_{k-1}, +\infty)$. В этом случае $p_i = F(a_i) - F(a_{i-1}); \ a_0 = -\infty, \ a_k = +\infty \ (i = 1, \dots, k)$. \\
Отметим, что $\sum_{i=1}^{k} p_i = 1$. Будем предполагать, что все $p_i > 0 \ (i = 1, \dots, k)$. \\
Пусть далее $n_1, n_2, \dots, n_k$ -- частоты  попадания выборочных элементов в подмножества $\Delta_1, \Delta_2, \dots, \Delta_k$ соответственно. \\
В случае справедливости гипотезы $H_0$ относительные частоты $\frac{n_i}{n}$ при большом $n$ должны быть близки к вероятностям $p_i \ (i = 1, \dots, k)$, поэтому за меру отклонения выборочного распределения от гипотетического с функцией $F(x)$ естественно выбрать величину
\begin{equation}
	Z = \sum_{i=1}^{k} c_i (\frac{n_i}{n} - p_i)^2 \text{,}
\end{equation}
где $c_i$ -- какие-нибудь положительные числа (веса). К.Пирсоном в качестве весов выбраны числа $c_i = \frac{n}{p_i} \ (i = 1, \dots, k)$. Тогда получается статистика критерия хи-квадрат К.Пирсона:
\begin{equation} \label{eq:chi_2}
	\chi^2 = \sum_{i=1}^{k} \frac{n}{p_i} (\frac{n_i}{n} - p_i)^2 = \sum_{i=1}^{k} \frac{(n_i - n p_i)^2}{n p_i},
\end{equation}
которая обозначена тем же символом, что и закон хи-квадрат. \\
К.Пирсоном доказана теорема об асимптотическом поведении статистики $\chi^2$, указывающая путь её применения. \\
\textbf{\emph{Теорема К.Пирсона.}} Статистика критерия $\chi^2$ асимптотически распределена по закону $\chi^2$ с $k - 1$ степенями свободы. \\
Это означает, что независимо от вида проверяемого распределения, т.е. функции $F(x)$, выборочная функция распределения статистики $\chi^2$ при $n \rightarrow \infty$ стремится к функции распределения случайной величины с плотностью вероятности:
\begin{equation}
	f_{k-1}(x) =
	\begin{cases}
		0, \ x \leq 0 \\ \\
		\dfrac{1}{2^{\frac{k - 1}{2}} \Gamma (\frac{k - 1}{2})} x^{\frac{k - 3}{2}} e^{-\frac{x}{2}}, \ x > 0
	\end{cases}
\end{equation}
Для прояснения сущности метода $\chi^2$ сделаем ряд замечаний. \\

\textbf{\textit{Замечание 1}}. Выбор подмножеств $\Delta_{1},\Delta_{2}, \dots, \Delta_{k}$ и их числа k в принципе ничем не регламентируется, так как $n \rightarrow \infty$. Но так как число $n$ хотя и очень большое, но конечное, то $k$ должно быть с ним согласовано. Обычно его берут таким же, как и для построения гистограммы, т.е. можно руководствоваться формулой
\begin{equation}
	k \approx 1.72\sqrt[3]{n}
	\label{eq:k_1}
\end{equation}
или формулой Старджесса
\begin{equation}
	k \approx 1 + 3.3 \lg n
\end{equation}
При этом, если  $\Delta_{1},\Delta_{2}, \dots ,\Delta_{k}$ — промежутки, то их длины удобно сделать равными, за исключением крайних -- полубесконечных.

\textbf{\textit{Замечание 2}}. (о числе степеней свободы).
Числом степеней свободы функции (по старой терминологии) называется число её независимых аргументов. Аргументами статистики $\chi^{2}$ являются частоты $n_{1},n_{2}, \dots,n_{k}$. Эти частоты связаны одним равенством $n_{1} + n_{2} + \dots + n_{k}  = n$, а в остальном независимы в силу независимости элементов выборки. Таким образом, функция $\chi^{2}$  имеет $k-1$ независимых аргументов: число частот минус одна связь. В силу теоремы Пирсона число степеней свободы статистики $\chi^{2}$  отражается на виде асимптотической плотности $f_{k - 1}(x)$.

На основе общей схемы проверки статистических гипотез сформулируем следующее правило. \\

\textbf{\textit{Правило проверки гипотезы о законе распределения по методу $\chi^{2}$}}.

\begin{enumerate}
	\item Выбираем уровень значимости $\alpha$.
	
	\item По таблице [?] находим квантиль $\chi^{2}_{1-\alpha}(k - 1)$ распределения хи-квадрат с $k-1$ степенями свободы порядка $1-\alpha$. 
	
	\item С помощью гипотетической функции распределения $F(x)$ вычисляем вероятности $p_{i} = P (X \in \Delta_{i})$, $i = 1, \dots, k$.
	
	\item Находим частоты $n_{i}$ попадания элементов выборки в подмножества $\Delta_{i}$, $i = 1, \dots, k$.
	
	\item Вычисляем выборочное значение статистики критерия $\chi^{2}$:
	\begin{equation}
		\chi^{2}_{B} =\sum_{i = 1}^{k}{\frac{(n_{i} - np_{i})^{2}}{np_{i}}}.
		\label{eq:chi_B}
	\end{equation}

	\item Сравниваем $\chi^{2}_{B}$ и квантиль $\chi^{2}_{1-\alpha}(k-1)$.
	\begin{itemize}
		\item Если $\chi^{2}_{B}$ < $\chi^{2}_{1-\alpha}$(k $-$ 1), то гипотеза $H_{0}$ на данном этапе проверки принимается.
		
		\item Если $\chi^{2}_{B} >= \chi^{2}_{1-\alpha}(k -1)$, то гипотеза $H_{0}$ отвергается, выбирается одно из альтернативных распределений, и процедура проверки повторяется.
	\end{itemize}
\end{enumerate} 

\textbf{\textit{Замечание 3}}. Из формулы (\ref{eq:chi_2}) видим, что веса $c_i = n/p_{i}$ пропорциональны $n$, т.е. с ростом $n$ увеличиваются. Отсюда следует, что если выдвинутая гипотеза неверна, то относительные частоты $n_{i}/n$ не будут близки к вероятностям $p_{i}$, и с ростом $n$ величина  $\chi^{2}_{B}$  будет увеличиваться. При фиксированном уровне значимости $\alpha$ будет фиксировано пороговое число - квантиль $\chi^{2}_{1-\alpha}(k-1)$, поэтому, увеличивая $n$, мы придём к неравенству $\chi^{2}_{B} > \chi^{2}_{1-\alpha}(k-1)$, т.е. с увеличением объёма выборки неверная гипотеза будет отвергнута.

Отсюда следует, что при сомнительной ситуации, когда $\chi^{2}_{B} \approx \chi^{2}_{1-\alpha}(k-1)$, можно попытаться увеличить объём выборки (например, в 2 раза), чтобы требуемое неравенство было более чётким.

\textbf{\textit{Замечание 4}}. Теория и практика применения критерия  $\chi^{2}$ указывают, что если для каких-либо подмножеств $\Delta_{i}$ $(i = 1, \dots ,k)$ условие $np_{i} \geq 5$ не выполняется, то следует объединить соседние подмножества (промежутки).

Это условие выдвигается требованием близости величин $\frac{(n_{i} -np_{i})}{\sqrt{np_{i}}}$, квадраты которых являются слагаемыми $\chi^{2}$  к нормальным $N(0,1)$. Тогда случайная величина в формуле (\ref{eq:chi_2}) будет распределена по закону, близкому к хи-квадрат. Такая близость обеспечивается достаточной численностью элементов в подмножествах $\Delta_{i}$.

\subsection{Доверительные интервалы для параметров нормального распределения}

\subsubsection{Доверительный интервал для математического ожидания $m$ нормального распределения}
Дана выборка ($x_{1}, x_{2}, \dots, x_{n}$) объёма $n$ из нормальной генеральной совокупности. На её основе строим выборочное среднее $\bar{x}$ и выборочное среднее квадратическое отклонение $s$. Параметры $m$ и $\sigma$ нормального распределения неизвестны.
\newline
Доказано, что случайная величина
\begin{equation}
	T = \sqrt{n - 1} \ \frac{\bar{x} - m}{s}
	\label{eq:T}
\end{equation}
называемая \textit{статистикой Стьюдента}, распределена по закону Стьюдента с $n-1$ степенями свободы. Пусть $f_{T}(x)$ -- плотность вероятности этого распределения. Тогда 
\begin{multline}
	P\left(-x < \sqrt{n - 1} \ \frac{\bar{x} - m}{s} < x \right) = 
	P\left(-x < \sqrt{n - 1} \ \frac{m - \bar{x}}{s} < x \right) = \\\
	= \int_{-x}^{x}{f_{T}(t)d t} = 2 \int_{0}^{x}{f_{T}(t)d t} = 
	2\left(  \int_{-\infty}^{x}{f_{T}(t)d t} - \frac{1}{2} \right) = 2F_{T}(x) - 1
	\label{eq:P_f_t}
\end{multline}
Здесь $F_{T}(x)$ — \textit{функция распределения Стьюдента с $n-1$ степенями свободы}.
\newline
Полагаем $2F_{T}(x)-1 = 1-\alpha$, где $\alpha$ — выбранный уровень значимости. Тогда $F_{T}(x) = 1-\alpha/2$. Пусть $t_{1-\alpha/2} (n-1)$ — квантиль распределения Стьюдента с $n-1$ степенями свободы и порядка $1-\alpha/2$. Из предыдущих равенств мы получаем 
\begin{equation}
	\begin{split}
		P\left(\bar{x} - \frac{sx}{\sqrt{n-1}} < m <  \bar{x} + \frac{sx}{\sqrt{n-1}}\right) = 2F_{T}(x) - 1 = 1 - \alpha,  \\
		P\left(\bar{x} - \frac{s t_{1-\alpha/2} (n-1)}{\sqrt{n-1}} < m <  \bar{x} + \frac{s t_{1-\alpha/2} (n-1)}{\sqrt{n-1}}\right)= 1 - \alpha, 
		\label{eq:P_m}     
	\end{split}
\end{equation}
что и даёт доверительный интервал для $m$ с доверительной вероятностью $\gamma = 1-\alpha$ [?].

\subsubsection{Доверительный интервал для среднего квадратического отклонения $\sigma$ нормального распределения}
Дана выборка ($x_{1},x_{2}, \dots ,x_{n}$) объёма $n$ из нормальной генеральной совокупности. На её основе строим выборочную дисперсию $s^{2}$. Параметры $m$ и $\sigma$ нормального распределения неизвестны. Доказано, что случайная величина $ns^{2}/\sigma^{2}$ распределена по закону $\chi^{2}$ с $n-1$ степенями свободы.
\newline
Задаёмся уровнем значимости $\alpha$ и находим квантили $\chi^{2}_{\alpha/2}(n-1)$ и $\chi^{2}_{1-\alpha/2}(n-1)$.
\newline
Это значит, что 
\begin{equation}
	\begin{split}
		P\left(\chi^{2}(n-1) < \chi^{2}_{\alpha/2}(n-1)\right) = \alpha/2, \\
		P\left(\chi^{2}(n-1) < \chi^{2}_{1-\alpha/2}(n-1)\right) = 1-\alpha/2
	\end{split}
	\label{eq:P_chi_2x2}        
\end{equation}
Тогда
\begin{multline}
	P\left(\chi^{2}_{\alpha/2}(n-1) < \chi^{2}(n-1) < \chi^{2}_{1-\alpha/2}(n-1)\right) = \\\ =
	P\left(\chi^{2}(n-1) < \chi^{2}_{1-\alpha/2}(n-1)\right) -P\left(\chi^{2}(n-1) < \chi^{2}_{\alpha/2}(n-1)\right) = \\\ = 1 - \alpha/2 -\alpha/2 = 1 - \alpha
	\label{eq:P_chi_2}
\end{multline}
Отсюда
\begin{multline}
	P\left(\chi^{2}_{\alpha/2}(n-1) < \frac{ns^{2}}{\sigma^{2}} < \chi^{2}_{1-\alpha/2}(n-1)\right) =
	P\left(\frac{1}{\chi^{2}_{1-\alpha/2}(n-1)} < \frac{\sigma^{2}}{ns^{2}} < \frac{1}{\chi^{2}_{\alpha/2}(n-1)} \right) = \\\ =
	P\left(\frac{s\sqrt{n}}{\sqrt{\chi^{2}_{1-\alpha/2}(n-1)}} < \sigma <  \frac{s\sqrt{n}}{\sqrt{\chi^{2}_{\alpha/2}(n-1)}}\right) = 1- \alpha
	\label{eq:interv}
\end{multline}
Окончательно
\begin{equation}
	P\left(\frac{s\sqrt{n}}{\sqrt{\chi^{2}_{1-\alpha/2}(n-1)}} < \sigma <  \frac{s\sqrt{n}}{\sqrt{\chi^{2}_{\alpha/2}(n-1)}}\right) = 1- \alpha,
	\label{eq:fin_interval}
\end{equation}
что и даёт доверительный интервал для $\sigma$ с доверительной вероятностью $\gamma = 1 - \alpha$ [?].

\subsection{Доверительные интервалы для математического ожидания $m$ и среднего квадратического отклонения $\sigma$ произвольного распределения при большом объёме выборки. Асимптотический подход}
При большом объёме выборки для построения доверительных интервалов может быть использован асимптотический метод на основе центральной предельной теоремы.

\subsubsection{Доверительный интервал для математического ожидания $m$ произвольной генеральной совокупности при большом объёме выборки}
Выборочное среднее $\bar{x} = \frac{1}{n}\sum_{i = 1}^{n}{x_{i}}$ при большом объёме выборки является суммой большого числа взаимно независимых одинаково распределённых случайных величин. Предполагаем, что исследуемое генеральное распределение имеет конечные математическое ожидание $m$ и дисперсию $\sigma^{2}$. Тогда в силу центральной предельной теоремы центрированная и нормированная случайная величина $\dfrac{\bar{x} - M\bar{x}}{\sqrt{D\bar{x}}} = \dfrac{\sqrt{n}·(\bar{x}-m)}{\sigma}$ распределена приблизительно нормально с параметрами 0 и 1. Пусть
\begin{equation}
	\Phi(x) = \frac{1}{2\pi}\int_{-\infty}^{x}{e^{-\frac{t^{2}}{2}}dt}
	\label{eq:f_lapl}
\end{equation}
- \textit{функция Лапласа}. Тогда
\begin{multline}
	P\left(-x < \sqrt{n} \ \frac{\bar{x} - m}{\sigma} < x \right) = 
	P\left(-x < \sqrt{n} \ \frac{m - \bar{x}}{\sigma} < x \right) \approx \\\
	\approx \Phi(x) - \Phi(-x)=\Phi(x) - [1 - \Phi(x)] = 2\Phi(x) - 1
	\label{eq:P_PHI}
\end{multline}
Отсюда
\begin{equation}
	P\left(\bar{x} - \frac{\sigma x}{\sqrt{n}} < m < \bar{x} - \frac{\sigma x}{\sqrt{n}} \right) \approx 2\Phi(x) - 1
	\label{eq:P_fin_PHI}
\end{equation}
Полагаем $2\Phi(x) - 1 = \gamma = 1 - \alpha$, тогда $\Phi(x) = 1 - \alpha/2$. Пусть $u_{1-\alpha/2}$ — квантиль нормального распределения $N(0,1)$ порядка $1-\alpha/2$. Заменяя в равенстве (\ref{eq:P_fin_PHI}) $\sigma$ на $s$, запишем его в виде
\begin{equation}
	P\left(\bar{x} - \frac{su_{1-\alpha/2}}{\sqrt{n}} < m < \bar{x} - \frac{su_{1-\alpha/2}}{\sqrt{n}} \right) \approx \gamma,
	\label{eq:P_fin_u}
\end{equation}
что и даёт доверительный интервал для $m$ с доверительной вероятностью $\gamma = 1-\alpha$ [?].

\subsubsection{Доверительный интервал для среднего квадратического отклонения $\sigma$ произвольной генеральной совокупности при большом объёме выборки}
Выборочная дисперсия $s^{2} = \sum_{i = 1}^{n}{\frac{(x_{i} - \bar{x})^{2}}{n}}$ при большом объёме выборки является суммой большого числа практически взаимно независимых случайных величин (имеется одна связь $\sum_{i=1}^{n}{x_{i}} = n\bar{x}$, которой при большом $n$ можно пренебречь). Предполагаем, что исследуемая генеральная совокупность имеет конечные первые четыре момента.
\newline
В силу центральной предельной теоремы центрированная и нормированная случайная величина $\frac{(s^{2}-Ms^{2})}{\sqrt{D s^{2}}}$ при большом объёме выборки $n$ распределена приблизительно нормально с параметрами 0 и 1. Пусть $\Phi(x)$ — функция Лапласа (\ref{eq:f_lapl}). Тогда
\begin{equation}
	P\left(-x < \frac{s^{2}-Ms^{2}}{\sqrt{D s^{2}}} < x\right)
	\approx \Phi(x) - \Phi(-x)=\Phi(x) - [1 - \Phi(x)] = 2\Phi(x) - 1
	\label{eq:P_as_sigma}
\end{equation}
Положим $2\Phi(x)-1 = \gamma = 1-\alpha$. Тогда $\Phi(x) = 1-\alpha/2$. Пусть $u_{1-\alpha/2}$ — корень этого уравнения — квантиль нормального распределения $N(0,1)$ порядка $1-\alpha/2$. Известно, что $Ms^{2} = \sigma^{2} -\frac{\sigma^{2}}{n} \approx \sigma^{2} \text{ и } D s^{2} = \frac{\mu_{4} -\mu_{2}^{2}}{n} + o(\frac{1}{n}) \approx \frac{\mu_{4} -\mu_{2}^{2}}{n}$ . Здесь $\mu_{k}$ — центральный момент $k$-го порядка генерального распределения; $\mu_{2} = \sigma^{2}$; $\mu_{4} = M[(x-M x)^{4}]$︀; $o(\frac{1}{n})$ — бесконечно малая высшего порядка, чем $\frac{1}{n}$, при $n\rightarrow \infty$. Итак, $D s^{2} \approx \frac{\mu_{4} -\mu_{2}^{2}}{n}$. Отсюда
\begin{equation}
	D s^{2} \approx \frac{\sigma^{4}}{n}(\frac{\mu_{4}}{\sigma^{4}} - 1) = 
	\frac{\sigma^{4}}{n}((\frac{\mu_{4}}{\sigma^{4}} - 3) + 2) = \frac{\sigma^{4}}{n}(E + 2) \approx \frac{\sigma^{4}}{n}(e + 2),
	\label{eq:Ds_2}
\end{equation}
где E = $\frac{\mu_{4}}{\sigma^{4}} - 3$ -- эксцесс генерального распределения, e = $\frac{m_{4}}{s^{4}} - 3$ -- выборочный эксцесс; $m_{4} = \frac{1}{n}\sum_{i =1}^{n}{(x_{i} - \bar{x})^{4}}$ -- четвёртый выборочный центральный момент. Далее,
\begin{equation}
	\sqrt{D s^{2}} \approx \frac{\sigma^{2}}{\sqrt{n}}\sqrt{e + 2}
	\label{eq:sqrt_Ds}
\end{equation}
Преобразуем неравенства, стоящие под знаком вероятности в формуле
\newline
$P\left(-x < \frac{s^{2}-Ms^{2}}{\sqrt{D s^{2}}} < x\right) = \gamma$:
\begin{equation}
	\begin{split}
		-\sigma^{2}U < s^{2} -\sigma^{2} < \sigma^{2}U; \\
		\sigma^{2}(1-U) < s^{2} < \sigma^{2}(1 + U); \\
		\frac{1}{\sigma^{2}(1 + U)} < \frac{1}{s^{2}} < \frac{1}{\sigma^{2}(1-U)};\\
		\frac{s^{2}}{1 + U} < \sigma^{2} < \frac{s^{2}}{1 - U};\\
		s(1 + U)^{-\frac{1}{2}} < \sigma < s(1-U)^{-\frac{1}{2}},
	\end{split}
	\label{eq:multi_ineq}
\end{equation}
где $U = u_{1-\alpha/2}︀\sqrt{\frac{e + 2}{n}}$ или
\newline
$s(1 +  u_{1-\alpha/2}︀\sqrt{\frac{e + 2}{n}})^{-\frac{1}{2}} <\sigma < s(1-u_{1-\alpha/2}︀\sqrt{\frac{e + 2}{n}})^{-\frac{1}{2}}$.
\newline
Разлагая функции в биномиальный ряд и оставляя первые два члена, получим
\begin{equation}
	s(1-0.5U) < \sigma < s(1 + 0.5U)
	\label{eq:s_U}
\end{equation}
или
\begin{equation}
	s(1-0.5u_{1-\alpha/2}︀\sqrt{\frac{e + 2}{n}}) < \sigma < s(1 + 0.5 u_{1-\alpha/2}︀\sqrt{\frac{e + 2}{n}})
	\label{eq:s_u}
\end{equation}
Формулы (\ref{eq:multi_ineq}) или (\ref{eq:s_u}) дают доверительный интервал для $\sigma$ с доверительной вероятностью $\gamma = 1-\alpha$. 
\newline
\textit{Замечание.} Вычисления по формуле (\ref{eq:multi_ineq}) дают более надёжный результат, так как в ней меньше грубых приближений.

\newpage
