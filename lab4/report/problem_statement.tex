\section{Постановка задачи}

Имеется выборка данных с интервальной неопределенностью. Число отсчетов в выборке равно 200. Используется модель данных с  уравновешенным интервалом погрешности. \\

$\bm{x} = \ \stackrel{\circ}{x} + \ \boldsymbol{\bm{\epsilon}}$; \quad $\boldsymbol{\bm{\epsilon}} = [-\epsilon, \epsilon]$  для некоторого $\epsilon >0 $, \\


Здесь $\stackrel{\circ}{x}$ -- данные некоторого прибора, $\epsilon = 10 ^ {-4}$ -- погрешность прибора.

Необходимо \cite{b:task}:
\begin{itemize}
	\item Иллюстрировать данные выборки
	\item Построить диаграмму рассеяния
	\item Построить линейную регрессионную зависимость варьированием неопределенности изменений с расширением и без сужения интервалов
	\item Построить линейную регрессионную зависимость варьированием неопределенности изменений с расширением и сужением интервалов
	\item Произвести анализ регрессионных остатков
	\item Построить информационное множество по модели
	\item Проиллюстрирвоать коридор совместных зависимостей
	\item Построить прогноз вне области данных
\end{itemize}

Файл с данными интервальной выборки "Channel\_1\_700nm\_0.2.csv" \quad расположен по следующей ссылке: \href{https://github.com/krzhld/mathstat/tree/lab4/lab4/source}{\textbf{исходные данные}} \\
Данные взяты из архива, расположенного по следующей ссылке: \href{https://github.com/AlexanderBazhenov/Solar-Data/blob/main/%D0%A1%D1%82%D0%B0%D1%82%D0%B8%D1%81%D1%82%D0%B8%D0%BA%D0%B0%20%D0%B8%D0%B7%D0%BC%D0%B5%D1%80%D0%B5%D0%BD%D0%B8%D0%B9.rar}{\textbf{архив с данными интервальных выборок}} (название использованного файла "Канал 1\_700nm\_0.2.csv")


\newpage
