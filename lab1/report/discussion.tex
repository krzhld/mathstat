\section{Обсуждение}

\subsection{Гистограмма и график плотности распределения}
По итогам проделанной работы можем сказать, что чем больше выборка для каждого из распределений, тем ближе ее гистограмма к графику плотности вероятности того закона, по которому распределены величины сгенерированной выборки. Это наглядно показывает гистограммы большой выборки ($n=10$) стандартного нормального распределения. \\
Чем меньше выборка, тем хуже по ней определяется характер распределения величины. Действительно, гистограммы по маленькой выборке ($n=10$) различных распределений очень похожи друг на друга (например нормальное распределение и распределение Коши).

\subsection{Характеристики положения и рассеяния}

Проанализировав данные из таблиц, можно судить о том, что дисперсия характеристик рассеяния для распределения Коши является некой аномалией: значения слишком большие даже при увеличении размера выборки - понятно, что это результат выбросов, которые мы могли наблюдать в результатах предыдущего задания.

\subsection{Доля и теоретическая вероятность выбросов}

По данным, приведенным в таблице, можно сказать, что чем больше выборка, тем ближе доля выбросов будет к теоретической оценке. Снова доля
выбросов для распределения Коши значительно выше, чем для остальных распределений. Равномерное распределение же в точности повторяет теоретическую оценку - выбросов мы не получали. \\
Боксплоты Тьюки действительно позволяют более наглядно и с меньшими усилиями оценивать важные характеристики распределений. Так, исходя из полученных рисунков, наглядно видно то, что мы довольно трудоёмко анализировали в предыдущих частях. 

\subsection{Эмпирическая функция и ядерные оценки плотности распределения}

Можем наблюдать на иллюстрациях с э. ф. р., что ступенчатая эмпирическая функция распределения тем лучше приближает функцию распределения реальной выборки, чем мощнее эта выборка. Заметим так же, что для распределения Пуассона и равномерного распределения отклонение функций друг от друга наибольшее. \\

Рисунки, посвященные ядерным оценкам, иллюстрируют сближение ядерной оценки и функции плотности вероятности для всех $h$ с ростом размера выборки. Для распределения Пуассона наиболее ярко видно, как сглаживает отклонения увеличение параметра сглаживания $h$. \\

В зависимости от особенностей распределений для их описания лучше подходят разные параметры $h$ в ядерной оценке: для равномерного распределения и распределения Пуассона лучше подойдет параметр $h=2h_{n}$, для распределения Лапласа − $h=h_{n}/2$, а для нормального и Коши − $h=h_{n}$. Такие значения дают вид ядерной оценки наиболее близкий к плотности, характерной данным распределениям. \\

Также можно увидеть, что чем больше коэффициент при параметре сглаживания $h_{n}$, тем меньше изменений знака производной у аппроксимирующей функции, вплоть до того, что при $h=2h_{n}$ функция становится унимодальной на рассматриваемом промежутке. Также видно, что при $h=2h_{n}$ по полученным приближениям становится сложно сказать плотность вероятности какого распределения они должны повторять, так как они очень похожи между собой.

\newpage
