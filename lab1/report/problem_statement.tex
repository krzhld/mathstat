\section{Постановка задачи}

Для 5 распределений:

\begin{itemize}
	\item Нормальное распределение $N(x,0,1)$
	\item Распределение Коши $C(x,0,1)$	
	\item Распределение Лапласа $L(x,0,\frac{1}{\sqrt{2}})$	
	\item Распределение Пуассона $P(k,10)$	
	\item Равномерное распределение $U(x,-\sqrt{3},\sqrt{3})$
\end{itemize}

\begin{enumerate}
	\item Сгенерировать выборки размером 10, 50 и 1000 элементов. \\
	Построить на одном рисунке гистограмму и график плотности распределения.
	
	\item Сгенерировать выборки размером 10, 100 и 1000 элементов. \\
	Для каждой выборки вычислить следующие статистические характеристики положения данных: $\overline{x}$, $med \ x$, $z_{R}$, $z_{Q}$, $z_{tr}$. Повторить такие вычисления 1000 раз для каждой выборки и найти среднее характеристик положения и их квадратов:
	\begin{equation}
		E(z)=\overline{z}
	\end{equation}
	Вычислить оценку дисперсии по формуле: 
	\begin{equation}
		D(z)=\overline{z^{2}}-\overline{z}^{2}
	\end{equation}
	Представить полученные данные в виде таблиц.
	
	\item Сгенерировать выборки размером 20 и 100 элементов. \\
	Построить для них боксплот Тьюки. \\
	Для каждого распределения определить долю выбросов экспериментально (сгенерировав выборку, соответствующую распределению, 1000 раз и вычислив среднюю долю выбросов) и сравнить с результатами, полученными теоретически.
	
	\item Сгенерировать выборки размером 20, 60 и 100 элементов. \\
	Построить на них эмпирические функции распределения и ядерные оценки плотности распределения на отрезке $[-4;4]$ для непрерывных распределений и на отрезке $[6;14]$ для распределения Пуассона.
\end{enumerate}

\newpage
