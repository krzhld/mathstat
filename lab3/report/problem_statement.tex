\section{Постановка задачи}

Имеется выборка данных с интервальной неопределенностью. Число отсчетов в выборке равно 200. Используется модель данных с  уравновешенным интервалом погрешности. \\

$\boldsymbol{\bm{x}} = \ \stackrel{\circ}{x} + \ \boldsymbol{\bm{\epsilon}}$; \quad $\boldsymbol{\bm{\epsilon}} = [-\epsilon, \epsilon]$  для некоторого $\epsilon >0 $, \\

Здесь $\stackrel{\circ}{x}$ -- данные некоторого прибора, $\epsilon = 10 ^ {-4}$ -- погрешность прибора. \par

Необходимо:
\begin{itemize}
	\item Иллюстрировать данные выборки
	\item Построить диаграмму рассеяния
	\item Найти базовые оценки исходной выборки
	\item Найти моду выборки и максимальную клику
	\item Произвести варьирование неопределенности измерений
	\item Вычислить меру совместности по индексу Жакара и относительную ширину моды
\end{itemize}

\newpage
