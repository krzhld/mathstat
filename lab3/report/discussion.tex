\section{Обсуждение}

\subsection{Множители коррекции $w$}
По преобладанию множителя 1 на гистограммах значений множителей коррекции (рис. \ref{pic:whysts}), можно сказать что примерно половина данных не требует коррекции. Этот факт свидетельствует о том, что линейная модель дрейфа данных является разумным приближением.

\subsection{Коэффициент Жаккара}
На рис. \ref{pic:jaccar} видно, что оптимальным множителем $R_{21}$ является число, равное 1.091266955. Однако видно, что коэффициент Жаккара при оптимальном значении едва-едва превышает 0, а интервал, при котором JK \geq 0, соизмерим с точкой (длина интервала оценивается $10^{-9} - 10^{-10} $). Это показывает на то, что исходные данные имеют ряд неточностей, которые сложно устранить. Это же можно было и заметить на рис. \ref{pic:intervals}, иллюстрирующий входные данные. Однако, как будет далее видно, подобранный коэффициент $R_{21}$ приблизит данные первого фотоприемника к данным второго фотоприемника.

\subsection{Гистограмма объединённых данных при оптимальном значении $R$}
Сравнивая гистограмму объединённых данных при оптимальном значении $R_{21}$ (рис. \ref{pic:union}) с гистограммами скорректированных данных (рис. \ref{pic:corrected}), видно, что гистограмма объединённых данных повторяет форму гистограммы входных данных с ФП1, однако пик гистограммы смещен в сторону значения пика на гистограмме ФП2.

\newpage
